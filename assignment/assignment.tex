\documentclass[a4paper,11pt]{article}

\usepackage{amsthm}
\newtheorem{definition}{Definition}

\usepackage{amstext}
\usepackage{amsfonts}

\usepackage[backend=biber]{biblatex}
\bibliography{assignment.bib}

\usepackage{hyperref}

\usepackage{fancyhdr}
\pagestyle{fancy}
\fancyhf{}
\rhead{Dominic Moylett - dm1905@my.bristol.ac.uk}
\cfoot{Page \thepage}

\title{COMSM0007 Assignment:\\A Summary of Public Key Encryption with keyword Search}
\author{Dominic Moylett - dm1905@my.bristol.ac.uk}

\begin{document}
    \maketitle

    \section{Introduction}

    Public Key Encryption with keyword Search\cite{cryptoeprint:2003:195} is a paper by Dan Boneh, Giovanni Di Crescenzo, Rafail Ostrovsky and Giuseppe Persiano published at Eurocrypt 2004. This paper looks at the ability to search for items contained within encrypted data.

    An example use is an email server: Alice can grant the server the ability to check if the email has the keyword 'Urgent'. When the server now receives an email, it checks if it has the keyword 'Urgent': if so the email is routed to Alice's pager, otherwise it is sent to her laptop. Non-interactive public key encryption with keyword search (sometimes abbreviated to 'searchable encryption') schemes mean that the server can check an encrypted email for keywords without learning what the keywords themselves are.

    Due to the computational cost of public key encryption in practice, a searchable encryption scheme does not allow a computer to search the entire encrypted message for keywords. Instead, Alice gives the machine is given access to a list of keywords that they are able to search for. When Bob sends an email, he then selects a number of words to be keywords, and sends them using a \textit{Public-Key Encryption with keyword Search} ($\mathsf{PEKS}$) of each keyword. When the server receives the email, it checks if any of the keywords Alice has given it the ability to search for are within the list of keywords sent by Bob.

    \section{Definitions}

    \begin{definition}
        A Non-interactive public key encryption with keyword search scheme consists of the following polynomial time randomised algorithms:

        \begin{enumerate}
            \item $\mathsf{KeyGen}(s)$: Given a security parameter $s$, produce public and private keys $A_{pub}, A_{priv}$ respectively.
            \item $\mathsf{PEKS}(A_{pub}, W)$: Given a public key $A_{pub}$ and keyword $W$, produce a searchable encryption of $W$.
            \item $\mathsf{Trapdoor}(A_{priv}, W)$: Given a private key $A_{priv}$ and keyword $W$, produce a trapdoor $T_W$.
            \item $\mathsf{Test}(A_{pub}, S, T_W)$: Given a public key $A_{pub}$, searchable encryption $S = \mathsf{PEKS}(A_{pub}, W')$ and trapdoor $T_W = \mathsf{Trapdoor}(A_{priv}, W)$, output 'yes' if $W = W'$ and 'no' otherwise.
        \end{enumerate}
    \end{definition}

    When Bob sends an email $msg$ with keywords $W_1,...,W_k$ to Alice, he computes the searchable encryptions for each keyword and sends the following:

    $$[E_{A_{pub}}[msg], \mathsf{PEKS}(A_{pub}, W_1),...,\mathsf{PEKS}(A_{pub}, W_k)]$$

    The email is received by Alice's server, which has trapdoors $T_{W'_1},...,T_{W'_l}$ where $T_{W'} = \mathsf{Trapdoor}(A_{priv}, W')$ for $W'$ in keywords $[W'_1,...,W'_l]$. The server then uses $mathsf{Test}$ with the trapdoors it has been given to see if the email contains one of the keywords specified by Alice.

    Accompanying this definition is a security definition, specified as a $\mathsf{PEKS}$ Security game against an active attacker $\mathcal{A}$. The game is defined similarly to IND-CCA:

    \begin{enumerate}
        \item The challenger runs $\mathsf{KeyGen}(s)$ to generate $A_{pub}$ and $A_{priv}$. $A_{pub}$ is given to the attacker.
        \item The attacker can adaptively ask for the trapdoor $T_W$ of any keyword $W \in \{0,1\}*$ of his choice.
        \item At some point, the attacker $\mathcal{A}$ sends two keywords $W_0, W_1$. The only restriction is that the attacker must have not previously asked for the trapdoors $T_{W_0}$ or $T_{W_1}$. The challenger picks a random bit $b \in {0,1}$ and gives the attacker $C = \mathsf{PEKS}(A_{pub}, W_b)$. $C$ is the challenge $\mathsf{PEKS}$.
        \item The attacker can continue to ask for trapdoors $T_W$ for any $W$ as long as $W \neq W_0, W_1$.
        \item The attacker eventually outputs a response bit $b' \in \{0,1\}$ and wins if $b = b'$.
    \end{enumerate}

    The adversary's advantage for winning the game is defined as:

    $$Adv_{\mathcal{A}}(s) = |Pr[b' = b] - \frac{1}{2}|$$

    \begin{definition}
        A $\mathsf{PEKS}$ is semantically secure against an adaptive chosen keyword attack if for any polynomial time attacker $\mathcal{A}$ we have that $Adv_{\mathcal{A}}(s)$ is a negligible function.
    \end{definition}

    \section{Constructions}

    The remainder of the paper gives two constructions of public key searchable encryption schemes.

    The first is based on bilinear maps. Bilinear maps are functions $e:G_1 \times G_1 \to G_2$ for groups $G_1, G_2$ of prime order $p$ which are efficiently computable and satisfy the properties that for any integers $x, y \in [1,p]$ it holds that $e(g^x, g^y) = e(g, g)^{xy}$ and that if $g$ is a generator of $G_1$ then $e(g, g)$ is a generator of $G_2$. Using $e$ and two hash functions $H_1: \{0, 1\}* \to G_1$ and $H_2: G_2 \to \{0, 1\}^{\log p}$, a searchable encryption scheme can be devised.

    The scheme is proven semantically secure against an adaptive chosen keyword attack in the random oracle model. This is proven via a reduction to the Bilinear Diffie-Hellman Problem: if an adversary $\mathcal{A}$ can break the scheme with non-negligible advantage $\epsilon$, an adversary $\mathcal{B}$ can be constructed which, given $g, g^a, g^b, g^c \in G_1$ as input, can compute $e(g, g)^{abc} \in G_2$ with advantage $\frac{\epsilon}{eq_Tq_{H_2}}$, where $q_T$ and $q_{H_2}$ are the number of calls to $\mathsf{Trapdoor}$ and $H_2$ respectively.

    The second construction is based on any trapdoor permutation. A result from Bellare et al. states that given any trapdoor permutation, we can construct a source indistinguishable public key encryption scheme. A public key encryption scheme is source indistinguishable if, given a ciphertext, it is computationally difficult to determine the public key that encrypted it. The scheme works by $A_{pub}$ being a set of public keys - one for each keyword - and $A_{priv}$ being their corresponding private keys. The security of this scheme is shown by a reduction: If the keywords can be distinguished, the underlying encryption scheme is not source indistinguishable.

    The advantage of this second scheme is that it does not rely on random oracles. But it has a cost in that the public and private keys grow linearly in size as the number of the keywords increase. This is improved by re-using source indistinguishable keys, which can still offer security via cover-free families. The construction uses parameters $(d, t, k, q)$ where $k$ is the number of words in the dictionary, $t$ is an upper bound on the number of keyword trapdoors given to the server by Alice and $d$ and $q$ are two integers satisfying $q = \lceil \frac{d}{4t} \rceil$ and $d \leq 16t^2(1 + \frac{\log(k/2)}{\log 3})$ and is based on a lemma by Du and Hwang that there exists a deterministic algorithm that, for any $(d, t, k, q)$ satisfying the above bounds, produces a $q$-uniform $t$-cover-free family. A reduction shows that if the $\mathsf{PKES}$ scheme has an adversary with advantage $\epsilon$, then there exists an algorithm that breaks source indistinguishable public key encryption scheme and aborts with probability $\frac{1}{poly(t, q, d)}$. Running this algorithm repeatedly until it doesn't abort yields an expected polynomial time adversary with advantage $\frac{\epsilon}{q^2}$.

    \section{Importance and Impact}
    Boneh et al. were not the first people to investigate the topic of searchable encryption, which was instead done by Song, Wagner and Perrig\cite{848445}. But Song, Wagner and Perrig's research looked into private key searchable encryption, as opposed to public key. Boneh et al. were also the first to provide definitions of syntax and security for searchable encryption, whereas Song, Wagner and Perrig instead argued that their scheme was secure by proving it was computationally indistinguishable from randomly generated text.

    Boneh et al. were not the only people working on this subject at the time. A similar concept was developed by Goh\cite{cryptoeprint:2003:216} and submitted to the Cryptology ePrint Archive a matter of weeks later. The authors acknowledged each other's work in later versions of both papers.

    Work inspired by this paper has focused on two general themes. The first theme is improving the ideas proposed in the original paper. One example of this is Abdalla et al.\cite{abdalla:se-revisited}, who looked at the concept of consistency within the schemes, where if $\mathsf{Test}(m, C) = \text{'yes'}$ and $\mathsf{Test}(m', C) = \text{'yes'}$ then $m = m'$. Baek, Safavi-Naini and Susilo\cite{baek:se-revisited} improved the efficiency of the bilinear maps construction by removing the need for the trapdoors to be sent from Alice to the email server via a secure (encrypted and authenticated) channel. And both papers investigated extending schemes so that keywords and trapdoors are only temporary, and expire after a given time such as the end of the day. The benefit of public key encryption with temporary keyword search ($\mathsf{PETKS}$) schemes is that when a server is given a trapdoor for a keyword, it is not able to learn if previous ciphertexts contain that keyword.

    The second theme is investigating what other preficates can be performed on encrypted data. The above papers by Song et al. and Boneh et al. provided methods for testing equality predicates on encrypted data. Boneh and Waters\cite{boneh:predicate} expanded this by developing encryption schemes catering for three more predicates: Comparison queries , subset queries and conjunctive queries . Katz, Sahai and Waters\cite{katz:predicate} developed an encryption scheme that allowed the evaluation of inner products over $\mathbb{Z_N}$ for some large integer $N$, which in turn allows the evaluation of other predicates including polynomial equations and disjunctions.

    It is worth noting that while Public Key Encryption with keyword Search has had a large impact in the theory world, it has seen little utilisation in practice. In particular, I have been unable to find a single email system which implements a $\mathsf{PEKS}$ scheme, or even a paper with an empirical analysis of such a scheme. A possible reason for this might be because of the practicality of the constructions for this problem; we are unlikely to see data large enough for this to be beneficial sent via email.

    But not all is lost; since 2004 when the paper was first published, a new paradigm has appeared. Boneh et al.'s work has given rise to investigations around the topic of keyword searches on encrypted data which is stored on a remote server\cite{chang:remote}. In particular, keyword searches on data stored in the cloud is a growing trend, with research expaning to problems such as approximate keyword searching\cite{5462196} and searching files for multiple keywords\cite{6674958}. This area has seen more experimental analysis, due to the ability to search larger amounts of data than you would through on a typical email.

    \printbibliography

\end{document}
